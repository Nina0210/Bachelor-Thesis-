\section{Background and Related Work}
\subsection{The PageRank Algorithm}
% Describe how PageRank works: power iteration, damping factor, convergence.

\begin{itemize}
  \item explain PageRank in 2 sentences
  \item history of the algorithm
  \item how does it work? 
  \item explain different methods of PageRank (Pregel in GraphX, Approxminate PR, ...)
  \item applications of PageRank
\end{itemize}

PageRank is an algorithm that was originally introduced to measure the relevance of a website in the World Wide Web. It was created by Larry Page and Sergey Brin, the founders of Google, to optimize the search engine. The foundation of the algorithm is a graph where every node represents a website in the internet and an edge translates to a hyperlink on a website. To calculate the PageRank values, a transition matrix is constructed, in which each row represents a state, a website, and contains the probabilities of moving from one node to another. If a website has d outgoing links, then the probability of moving from the current page to another page is 1/d. The transition matrix describes the behavior of a "random surfer". The random surfer model describes the probability of a random user to visit a website. Finding the PageRank values is a Markov chain process, and its stationary distribution is the PageRank vector. Additionally, to make the Markov chain more realistic, a damping factor is introduced, which is commonly set to 0.85. That means with probability of $\alpha$ a random surfer clicks on an outgoing hyperlink on the current website and with probability of $1-\alpha$ the surfer "teleports" to a random website in the graph. This characteristic is important for PageRank because web graphs can have dangling nodes, disconnected parts or cycles. Thus, the damping factor ensures that the Markov chain is irreducible and aperiodic, which guarantees a convergence to a unique stationary distribution. 
The PageRank algorithm is defined as the following:
\begin{equation}
    \pi_v = \frac{1-\alpha}{n}+c\sum_{u\in N^-(v)}\frac{\pi_u}{d^+(u)} \quad\text{\cite{chebolu_pagerank_2008},}
\end{equation} 
where: 
\begin{itemize}
    \item $\pi_v$ is the PageRank of the node $v$
    \item $\alpha$ is the damping factor
    \item $n$ is the total number of nodes in the graph
    \item $N^-(v)$ is the set of ingoing edges of $v$
    \item $d^+(u)$ is the out degree of $u$
\end{itemize} 
% talk about tolerance and convergence
Each entry of the PageRank vector is the probability that a random surfer lands on the given node.

 
\subsection{Challenges in Large-scale Graph Processing}
% Memory and computational bottlenecks with big graphs.
\begin{itemize}
  \item also take parts of the introduction
\end{itemize}

\subsection{Apache Spark and GraphX}
% Explain architecture, vertex-centric computation, and limits of GraphX PageRank.
\begin{itemize}
    \item explain apache spark and graphx architecture
    \item explain memory management
    \item explain PageRank algorithm of GraphX
    \item explain RDD's
\end{itemize}

\subsection{Approximate PageRank Methods}
% Iteration, sampling, matrix approximation, and Monte Carlo approaches.
\begin{itemize}
    \item explain approximate pagerank methods also take parts of introduction
    \item explain how it's more memory efficient than original methods
\end{itemize}