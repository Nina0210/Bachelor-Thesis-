\section{Conclusion}
% Summary of your work and suggestions for future research (e.g., better partitioning, dynamic walker allocation).

The goal of this thesis was to analyze wether Monte Carlo PageRank is a serious alternative for standard PageRank methods such as GraphX PageRank implemetation. The central research question asked wether Monte Carlo PageRank is able to operate under constraint memory conditions with trading off only little accuracy and performance. The evaluation across real world and synthetic graphs proves that the approximate Monte Carlo method is better compared to standard iterative methods.\par

One of the key findings is that Monte Carlo PageRank successfully operates on only 450 MiB per Executor across all graph sizes and structures, while GraphX fails on low memory configurations or suffers from low performance even with the highest memory configurations of 8 GiB. Surprisingly the low memory configuration doesn't compromise performance and accuracy as expected at the beginning. On web graphs Monte Carlo PageRank keeps a stable runtime around 100 seconds while GraphX encounters significant performance issues on low memory configurations. Additionally the approximate approach achieves 75\%-90\% similarity depending on the walker configuration when comparing with the top 20 nodes of GraphX PageRank. Considering that GraphX is not able to operate successfully across all memory configurations, getting such an approximate ranking is an acceptable trade off. Moreover, the cost experiments show that Monte Carlo is less costly than GraphXs PageRank as it uses 5 to 10 times fewer resources. Monte Carlo PageRanks performance behavior also makes the method much more predictable.\par

However, Monte Carlo PageRank is not a universal solution. Applications that require exact rankings should still rely on iterative methods that guarantees a precise result. Another downside is the instability of results across experiments due to the randomness of Monte Carlo PageRank. Therefore the method requires multiple runs to get a stable ranking. Although the approximate method performs better in memory constraint environments GraphX may have lower runtimes When there is sufficient memory allocated.\par

Future work could evaluate the Monte Carlo method further on synthetic graphs of different sizes based on the number of edges instead of the nodes. Furthermore, it could be of interest to analyze the behavior of Monte Carlo PageRank on dynamic graphs. Finally, a system that decides for the user which method is best for the given graph and requirements could be developed. \par

Overall this thesis demonstrates that the approximate Monte Carlo PageRank method is a memory and cost efficient alternative to standard iterative PageRank methods. 